% "ИСР"

\documentclass[a4paper,12pt]{article} % тип документа

%  Русский язык
\usepackage[T2A]{fontenc}			% кодировка
\usepackage[utf8]{inputenc}		
\usepackage[english,russian]{babel}	
% Математика
\usepackage{amsmath,amsfonts,amssymb,amsthm,mathtools} 
\usepackage{wasysym}
\usepackage{hyperref} % гиперссылки
%Заговолок
\author{Нюхалов Денис}
\title{Cписок команд для создания таблиц в \LaTeX{}}
\date{Декабрь 2020}


\begin{document} % начало документа

\maketitle
\newpage

\section{Основные команды}

\begin{itemize}
    \item Таблицы создаются при помощи окружения tabular\\\\
    \textbackslash begin \{tabular\}[положение]\{столбцы\}\\ \textbackslash end\{tabular\}
    \item В tabular важную роль играет атрибут положение. Именно он определяет налчичие границ у таблицы, align к левой или правой ячейки таблицы
    \item Расположение в документе и заголовки задаются в окружении table\\\\
    \textbackslash begin\{table\} \textbackslash end\{tabular\}
    \item Таблица заполняется в соответсвии с указанным в окружении tabular положением. Для разделения элементов таблиц по столбцам используется амперсант (\&)\\\\
    \textbackslash begin \{tabular\}[положение]\{столбцы\}\\
    элемент \& элемент \& элемент\\
    \textbackslash end\{tabular\}
    \item Для перехода на следующую строку используется команда \textbackslash hline \\\\
    \textbackslash begin \{tabular\}[положение]\{столбцы\}\\
    элемент \& элемент \& элемент\\
    \textbackslash hline\\
    элемент \& элемент \& элемент\\
    \textbackslash end\{tabular\}
    
\end{itemize}
\end{document}