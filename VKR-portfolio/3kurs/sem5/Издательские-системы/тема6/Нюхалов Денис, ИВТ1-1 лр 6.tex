% "Лабораторная работа 2"

\documentclass[a4paper,12pt]{article} % тип документа

% report, book

%  Русский язык

\usepackage[T2A]{fontenc}			% кодировка
\usepackage[utf8]{inputenc}			% кодировка исходного текста
\usepackage[english,russian]{babel}	% локализация и переносы


% Математика
\usepackage{amsmath,amsfonts,amssymb,amsthm,mathtools} 


\usepackage{wasysym}

\usepackage{hyperref}

%Заговолок
\author{Нюхалов Денис}
\title{Лабораторная работа №6\\Работа с текстом в \LaTeX{}}
\date{Декабрь 2020}


\begin{document} % начало документа

\maketitle
\newpage
\begin{center}
    \section{Задание 1. Ответы на вопросы}
\end{center}
\begin{enumerate}
    \item Для чего предназначена издательская система \LaTeX{}?\\\\
    \textit{TeX — система компьютерной вёрстки. \LaTeX{}, же - это популярный набор макрорасширений (или макропакет) для системы TeX. Он облегчает набор сложных документов.}
    \item В каких случаях рационально её использовать?\\\\
    \textit{Она пойдет для документов, требующих сложного и не очень форматирования, с хорошей логической структурой.}  
    \item Какие преимущества имеет работа в этот системе?\\\\
    \textit{Большой выбор способов форматирования докумета. При успешном освоении можно быстро и эффективно создавать большие документы. Хороший инструментарий для технической литераутры, особенно для формул}
    \item Какие сложности могут возникнуть при работе в этот системе?\\\\
    \textit{Имеется порог вхождения. Для работы с системой \LaTeX{} необходимо освоить и запомнить некоторые команды и правила.В современных визуальных изд. системах такого нет. \LaTeX{} сложен для изучения начинающими}
    \item Какие недостатки отмечают пользователи при работе с этой системой?\\\\
    \textit{Пользователь при наборе исходного файла не видит на экране результат. Требуется компиляция исходного файла для его результата. Иногда на экране появляется вовсе не то, что хотелось бы видеть.}
    
\end{enumerate}
\newpage
\begin{center}
  \section{Задание 2. 10 вопросов к прочитанному тексту.}  
\end{center}
\begin{enumerate}
    \item Какие основные элементы создания логической структуры текста в \LaTeX{}?\\\\
    \textit{Абзац и предложение}
    \item Как использовать служебные символы в тексте?\\\\
    \textit{Их можно экранировать их с помощью символа "$\backslash$".}
    \item Какие существуют классы документов в \LaTeX{}\\\\
    \textit{Классы документов:\begin{itemize}
        \item article -- для статей, научных журналов, презентация, которотких отчётов, локументаций и др.
        \item report -- для более длинных отчётов, содержащих несколько глав, небольших книжек, диссертаций.
        \item book -- для настоящих книг
        \item slides -- для слайдов. Использует больште буквы без засечек.
    \end{itemize}
    }
    \item Что нужно для корректного подключения иностранного языка в \LaTeX{}?\\\\
    \textit{Все автогенерируемые строки\footnote{Содержание, список илюстраций, библиография...} должны быть переведены на этот язык. \LaTeX{} должен знать правила переноса языка. \LaTeX{} должен знать Специфичные для нового языка типографические правила.}
    \item Что такое babel?
    \textit{Это пакет, для перевода автоматически генерируемых текстовых строк}
    \item Какая команда подключает язык в \LaTeX{}?\\\\
    \textit{$\backslash$uselanguage[язык1, язык2]{babel}}
    \item Как создать оглавление в \LaTeX{}?\\\\
    \textit{Вызвать команду tableofcontents}
    \item Что такое хрупкая командая?\\\\
    \textit{Команды которые не работают в аргументах других команд.}
    \item Что такое хрупкая командая?\\\\
    \textit{Команды которые не работают в аргументах других команд.}
    \item Как защитить хрупкую команду?\\\\
    \textit{при помощи команды $\backslash$protect}
\end{enumerate}{}
\begin{center}
    \section{Источники}
\end{center}
\begin{enumerate}
    \item \href{https://ru.wikipedia.org/wiki/LaTeX#:~:text=LaTeX%20(%D0%BF%D1%80%D0%BE%D0%B8%D0%B7%D0%BD%D0%BE%D1%81%D0%B8%D1%82%D1%81%D1%8F%20%2F%CB%88l%C9%91%CB%90,LA%CE%A4%CE%95%CE%A7.}{Ссылка}
    \item \href{https://www.opennet.ru/docs/RUS/linux_base/node380.html}{Ссылка}
\end{enumerate}


\end{document} % конец документа