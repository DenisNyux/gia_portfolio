% "ИСР"

\documentclass[a4paper,12pt]{article} % тип документа

%  Русский язык
\usepackage[T2A]{fontenc}			% кодировка
\usepackage[utf8]{inputenc}		
\usepackage[english,russian]{babel}	
% Математика
\usepackage{amsmath,amsfonts,amssymb,amsthm,mathtools} 
\usepackage{wasysym}
\usepackage{hyperref} % гиперссылки
%Заговолок
\author{Нюхалов Денис}
\title{Команды для создания текстового документа в \LaTeX{}}
\date{Декабрь 2020}


\begin{document} % начало документа

\maketitle
\newpage

\section{Основные команды}

Самой первой командой является \textbackslash documentclass[]\{класс\}. Она задает стиль оформления: формат бумаги, шрифт и т.д.\\
Основных классов документов 4:
\begin{itemize}
        \item article -- для статей, научных журналов, презентация, которотких отчётов, локументаций и др.
        \item report -- для более длинных отчётов, содержащих несколько глав, небольших книжек, диссертаций.
        \item book -- для настоящих книг
        \item slides -- для слайдов. Использует большие буквы без засечек.
\end{itemize}
Тело документа находится в окружении \textbackslash begin\{document\} \textbackslash end\{document\}\\\\
Для создания разделов и подразделов документа используются команды \textbackslash section\{\} и \textbackslash subsection\{\}\\\\
Для переноса текста на следуюшую строку нужно использовать \textbackslash\textbackslash\\\\
Для перехода на следующий абзац нужно поставить пустую строку.\\\\
Для настройки абзацных отступов используется команда \textbackslash parindent=2cm\\\\
Для создания сноски используется команда \textbackslash footnote\{\}\\\\
Для изменения формата текста на 
\begin{itemize}
    \item Жирный шрифт - \textbackslash textbf\{\}
    \item Курсив - \textbackslash textit\{\}
    \item Подчеркнутый - \textbackslash underline\{\}
\end{itemize}

\end{document}