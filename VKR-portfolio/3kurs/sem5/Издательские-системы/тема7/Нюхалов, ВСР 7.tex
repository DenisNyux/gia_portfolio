\documentclass[a4paper,12pt]{article} % тип документа

%  Русский язык
\usepackage[T2A]{fontenc}			% кодировка
\usepackage[utf8]{inputenc}		
\usepackage[english,russian]{babel}	
% Математика
\usepackage{amsmath,amsfonts,amssymb,amsthm,mathtools} 
\usepackage{wasysym}
\usepackage{hyperref} % гиперссылки
%Заговолок
\author{Нюхалов Денис}
\title{Формулы сокращенного умножения}
\date{Декабрь 2020}
\begin{document}
\maketitle

Квадрат суммы (\ref{ssum}):
\begin{equation}
  {(a + b)}^2 = a^2 + 2ab + b^2
  \label{ssum}
\end{equation}

Квадрат разности (\ref{sdif}):
\begin{equation}
  {(a - b)}^2 = a^2 - 2ab + b^2
  \label{sdif}
\end{equation}

Разность квадратов (\ref{dsqr}):
\begin{equation}
  a^2 - b^2 = (a - b) (a + b)
  \label{dsqr}
\end{equation}

Куб суммы (\ref{csum}):
\begin{equation}
  {(a + b)}^3 = a^3 + 3a^2 b + 3a b^2 + b^3
  \label{csum}
\end{equation}

Куб разности (\ref{cdif}):
\begin{equation}
  {(a - b)}^3 = a^3 - 3a^2 b + 3a b^2 - b^3
  \label{cdif}
\end{equation}

Сумма кубов (\ref{scub}):
\begin{equation}
  a^3 + b^3 = (a + b) (a^2 - ab + b^2)
  \label{scub}
\end{equation}

Разность кубов (\ref{dcub}):
\begin{equation}
  a^3 - b^3 = (a - b) (a^2 + ab + b^2)
  \label{dcub}
\end{equation}


\end{document}
