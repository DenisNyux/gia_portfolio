% "Лабораторная работа 2"

\documentclass[a4paper,14pt]{article} % тип документа

% report, book

%  Русский язык

\usepackage[T2A]{fontenc}			% кодировка
\usepackage[utf8]{inputenc}			% кодировка исходного текста
\usepackage[english,russian]{babel}	% локализация и переносы


% Математика
\usepackage{amsmath,amsfonts,amssymb,amsthm,mathtools} 


\usepackage{wasysym}

\usepackage{hyperref}

%Заговолок
\author{Нюхалов Денис}
\title{Занятие 7. Работа на паре.}
\date{Декабрь 2020}


\begin{document} % начало документа
\maketitle
\newpage
\section{Изменения шрифта}


Стандарт\\
\tiny tiny\\
\small small\\
\scriptsize scriptsize\\
\footnotesize footnotesize\\
\normalsize normalsize\\
\large large\\
\Large Large\\
\LARGE LARGE\\
\huge Huge\\





\section{Гиперссылки}

Ссылка на moodle\\
\href{https://moodle.herzen.spb.ru/}{Moodle main page}

\section{Выравнивание}

\begin{flushright}
Выравнивание по правому краю
\end{flushright}

\begin{center}
Выравнивание по центру
\end{center}

\begin{flushleft}
Выравнивание по левому краю
\end{flushleft}

\section{Сноски}
Текст к сноске
\footnote{Сноска на что то}

\section{Формулы}
Формула в тексте $a+b$, сумма a и b\\
Формула на новой строчке $$a^2 + b^2 = c^2$$\\
Формула с ссылкой\\
\begin{equation}
    \label{pif}
    a^2 + b^2 = c^2
\end{equation}
Ссылка на формулу \eqref{pif} в тексте\\

\subsection{Верхние и нижние индексы в формулах}
$m_1$\\
$m_{12}$\\
$m^2$\\
$m^{22}$\\

\subsection{Стандартные функции в формулах}
$\sin(x)$\\
$\sin x$\\
$\arctg(x) = \sqrt(3)$\\
$\arctg(x) = \sqrt[5](3)$\\
$\log_{x-1}{x^2+5*x-4}$

\section{Функции}
Формула суммы некоторых значений $\sum_{i=1}^{n}a_i+b_i$ некоторых значений\\
\[ \sum_{i=1}^{n}a_i+b_i \]
\subsection{Интегралы}
$I=\int r^2dm$ интегрирование по dm.\\
$$I=\int r^2dm$$
$$I=\int_{0}^{1} r^2dm$$
\subsection{Знаки}
$$3\times2=2\cdot 3\\$$
$$1\dots n\\$$
$$\frac{2}{4} = \frac{1}{2}\\$$
$$\underbrace{1+2+3\dots+n}\\$$
\section{Диакритические знаки}
\subsection{Надстрочные знаки}
\[\dot{x}=0\]
\[\tilde{a}=\bar{b}\]
\[\widetilde{Sas}=\overline{saaas}\]
многоточие \cdots
\subsection{Векторы}
Вектор а имеет координаты (0,3,4)
\[\overrightarrow(a) = (0,3,4)\]
\[\overrightarrow(a) = \mathbf{a}\]
\subsection{Фигурная скобка}
\[\underbrace{1+2+\cdots+n}_{n} = N\]
\[\overbrace{1+2+\cdots+n}_{n} = N\]
\[\overbrace{1+2+\cdots+n}^{n} = N\]
\subsection{Написание условия перехода над знаком}
Команда \textbf{stackrel}\\
Например\\
\[(x-1)(x+1)>0 \stackrel{x>0}{\longrightarrow}(x-1)>0\]
\section{Буквы других алфавитов}
\[\sin \alpha=0\]
\[\omega=\frac{2\pi}{T}\]
непривычный вид \\
$\epsilon$ \\
$\phi$ \\
как в учебниках \\
$\varepsilon$ \\
$\varphi$
\subsection{Математические шрифты}
\textbf{mathbb}
\[x \in R\]
\[x \in \mathbb{R}\]

\end{document} % конец документа
