% "Лабораторная работа 7"

\documentclass[a4paper,12pt]{article} % тип документа

%  Русский язык
\usepackage[T2A]{fontenc}			% кодировка
\usepackage[utf8]{inputenc}			% кодировка исходного текста
\usepackage[english,russian]{babel}	% локализация и переносы
% Математика
\usepackage{amsmath,amsfonts,amssymb,amsthm,mathtools} 
\usepackage{wasysym}
\usepackage{hyperref} % гиперссылки
%Заговолок
\author{Нюхалов Денис}
\title{Лабораторная работа №7\\"Особенности технологии создания текста с формулами"}
\date{Декабрь 2020}


\begin{document} % начало документа

\maketitle
\newpage
\section{Формулы}

\begin{equation}
\int \frac{dx}{\ln x}=\ln|\ln x|+\sum_{i=1}^\infty \frac{(\ln x)^i}{i\cdot i!}
\end{equation}
\begin{equation}
\int \frac{dx}{(\ln x)^n}=-\frac{x}{(n-1)(\ln x)^{n-1}}+\frac{1}{n-1}\int \frac{dx}{(\ln x)^{n-1}}\text{ для } n\neq 1
\end{equation}
\begin{equation}
\int x^m \ln x dx=x^{m+1}\Bigg( \frac{\ln x}{m+1}-\frac{1}{(m+1)^2}\Bigg) \text{ для } m \neq -1
\end{equation}
\begin{equation}
\int x^m(\ln x)^n dx=\frac{x^{m+1}(\ln x)^n}{m+1}-\frac{n}{m+1}\int x^m(\ln x)^{n-1}dx \text{ для } m \neq -1
\end{equation}
\begin{equation}
\int\frac{(\ln x)^n dx}{x}=\frac{(\ln x)^{n+1}}{n+1} \text{ для } n\neq -1
\end{equation}
\begin{equation}
\int \frac{\ln x dx}{x^m}=-\frac{\ln x}{(m-1)x^{m-1}}-\frac{1}{(m-1)^2 x^{m-1}} \text{ для } m \neq 1
\end{equation}
\begin{equation}
\int \frac{(\ln x)^n dx}{x^m}=-\frac{(ln x)^n}{(m-1)x^{m-1}}+\frac{n}{m-1}\int\frac{(ln x)^{n-1}dx}{x^m} \text{ для } m\neq 1 
\end{equation}
\end{document}