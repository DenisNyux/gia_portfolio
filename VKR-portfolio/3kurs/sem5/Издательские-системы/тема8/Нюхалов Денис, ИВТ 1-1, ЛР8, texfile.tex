\documentclass[a4paper,12pt]{article} % тип документа


%  Русский язык
\usepackage[T2A]{fontenc}			% кодировка
\usepackage[utf8]{inputenc}			% кодировка исходного текста
\usepackage[english,russian]{babel}	% локализация и переносы
% Математика
\usepackage{amsmath,amsfonts,amssymb,amsthm,mathtools} 
\usepackage{wasysym}
\usepackage{hyperref} % гиперссылки
%Заговолок
\author{Нюхалов Денис}
\title{Лабораторная работа №8\\"Создание матриц средствами \LaTeX{}"}
\date{Декабрь 2020}


\begin{document} % начало документа

\maketitle
\newpage

\section{Пример 1. Умножение матрицы на число}

\textbf{Дано:} \\
Матрица
\begin{displaymath}
\mathbf{A} =
\left( \begin{array}{ccc}
1 & 2 & 3\\
4 & 5 & 6 
\end{array} \right)
\end{displaymath}
Число $k=2$.\\
\textbf{Найти:}\\
Произведение матрицы на число: $A\times k=B$\\
B -- ?\\
\textbf{Решение:}\\
Для того чтобы умножить матрицу А на число k нужно каждый элемент матрицы А умножить на это число.
Таким образом, произведение матрицы А на число k есть новая матрица:
\begin{displaymath}
B = 2\times A = 2\times
\begin{pmatrix}
1 & 2 & 3\\
4 & 5 & 6 
\end{pmatrix}= 
\begin{pmatrix}
2 & 4 & 6\\
8 & 10 & 12 
\end{pmatrix}
\end{displaymath}
Ответ:
\begin{displaymath}
B = \begin{pmatrix}
2 & 4 & 6\\
8 & 10 & 12 
\end{pmatrix}
\end{displaymath}

\section{Пример 2. Умножение матриц}
\textbf{Дано:} \\
Матрица
\begin{displaymath}
A=\begin{pmatrix}
2 & 3 & 1\\
-1 & 0 & 1 
\end{pmatrix}
B=\begin{pmatrix}
2 & 1\\
-1 & 1\\
3 & -2\\
\end{pmatrix}
\end{displaymath}
\textbf{Найти:}\\
Произвидение матриц: $A\times B = C$\\
C -- ?\\
\textbf{Решение:}\\
Каждый элемент матрицы $ C = A \times B$, расположенный в $i$-ой строке и $j$-ом столбе, равен сумме произведений элементов $i$-ой матрицы A на соответствующие элементы $j$-го столбца матрицы B. Строки матрицы A умножаем на столбцы матрицы B и получаем:
\begin{displaymath}
C=A\times B =\begin{pmatrix}
2 & 3 & 1\\
-1 & 0 & 1\\
\end{pmatrix}\times
\begin{pmatrix}
2 & 1\\
-1 & 1\\
3 & -2\\
\end{pmatrix}=
\end{displaymath}
\begin{displaymath}
=\begin{pmatrix}
2\times 2+3\times(-1)+1\times 3 & 2\times 1+3\times 1+1\times (-2)\\
-1\times 2+0\times(-1)+1\times 3 & -1\times 1+0\times 1+1\times (-2)\\
\end{pmatrix}
\end{displaymath}
\begin{displaymath}
C=A\times B =\begin{pmatrix}
4 & 3 \\
1 & -3 \\
\end{pmatrix}
\end{displaymath}
Ответ:\begin{displaymath}
C=\begin{pmatrix}
4 & 3 \\
1 & -3 \\
\end{pmatrix}
\end{displaymath}

\section{Пример 3. Транспонирование матрицы}
\textbf{Дано:} \\
Матрица
\begin{displaymath}
A=\begin{pmatrix}
7 & 8 & 9\\
1 & 2 & 3 
\end{pmatrix}
\end{displaymath}
\textbf{Найти:}\\
Найти матрицу транспонированную данной.\\
$A^T$ -- ?\\
\textbf{Решение:}\\
Транспонирование матрицы $А$ заключается в замене строк этой матрицы ее столбцами с сохранением номеров. Полученная матрица обозначается через $A^T$.
\begin{displaymath}
A=\begin{pmatrix}
7 & 8 & 9\\
1 & 2 & 3 
\end{pmatrix}\Rightarrow A^T= \begin{pmatrix}
7 & 1\\
8 & 2\\
9 & 3
\end{pmatrix}
\end{displaymath}
Ответ:\begin{displaymath}
A^T=
\begin{pmatrix}
7 & 1\\
8 & 2\\
9 & 3
\end{pmatrix}
\end{displaymath}

\section{Пример 4. Обратная матрица}
\textbf{Дано:} \\
Матрица
\begin{displaymath}
A=\begin{pmatrix}
2 & -1\\
3 & 1
\end{pmatrix}
\end{displaymath}
\textbf{Найти:}\\
Найти обратную матрицу для матрицы $A$.\\
$A^{-1}$ -- ?\\
\textbf{Решение:}\\
Находим $det A$ и проверяем $det A\ne 0$:
\begin{displaymath}
det A=\begin{vmatrix}
2 & -1\\
3 & 1
\end{vmatrix}=2\times 1-3\times(-1)=5
\end{displaymath}
$det A=5\ne 0.$\\
Составляем вспомогательную матрицу $a^V$ из алгебраических дополнений $A_{ij}$:
\begin{displaymath}
A^V=\begin{pmatrix}
1 & -3\\
1 & 2
\end{pmatrix}
\end{displaymath}
Транспонируем матрицу $A^V$:
\begin{displaymath}
(A^V)^T=\begin{pmatrix}
1 & 1\\
-3 & 2
\end{pmatrix}
\end{displaymath}
Каждый элемент полученной матрицы делим на $det A$:
\begin{displaymath}
A^{-1}=\frac{1}{detA}(A^V)^T=\frac{1}{5}\times \begin{pmatrix}
1 & 1\\
-3 & 2
\end{pmatrix}=\begin{pmatrix}
\frac{1}{5} & \frac{1}{5}\\
-\frac{3}{5} & \frac{2}{5}
\end{pmatrix}
\end{displaymath}
Ответ:\begin{displaymath}
A^{-1}=\begin{pmatrix}
\frac{1}{5} & \frac{1}{5}\\
-\frac{3}{5} & \frac{2}{5}
\end{pmatrix}
\end{displaymath}
\end{document}